\section*{Problemstellung}
Die Stadt Nemreb möchte sich auf künftige Pandemiesituationen besser vorbereiten und ein 
robustes, skalierbares System zur Impfterminregistrierung entwickeln. Ziel ist es, eine 
Webanwendung bereitzustellen, die auch bei hoher Nachfrage stabil und effizient bleibt. 
Die technische Umsetzung stellt verschiedene Herausforderungen, darunter die 
Benutzerverwaltung mit Registrierung und Anmeldung per E-Mail sowie die Terminbuchung, 
bei der Nutzer freie Termine einsehen, buchen und absagen können. Dabei müssen die 
Kapazitäten der Impfzentren berücksichtigt werden, sodass Zeitslots nicht überbucht werden.
Erweiterte Buchungsfunktionen sollen es ermöglichen, Termine auch für Familienmitglieder 
zu reservieren. Zudem wird für jede Buchung ein QR-Code generiert, der in einem 
Bestätigungspdf enthalten ist, wobei der Datenschutz gewahrt bleibt und keine 
persönlichen Daten im QR-Code gespeichert werden. Datenschutz und Sicherheit sind 
zentrale Aspekte des Systems, weshalb auf Transparenz, minimale Datenspeicherung und das 
Hosting in der behördlichen Infrastruktur Wert gelegt wird. Ein weiteres essenzielles 
Kriterium ist die Fehlertoleranz und Skalierbarkeit des Systems, um auch unter hoher Last 
performant zu bleiben und kurzfristige Anpassungen zu ermöglichen.
Zur Optimierung der Systemstabilität wird ein Echtzeit-Monitoring der Systemauslastung 
integriert, sodass auf Engpässe schnell reagiert werden kann. Das System soll als 
Open-Source-Projekt entwickelt werden und auf einem aktuellen Tomcat-Server mit einer 
Open-Source-Datenbank unter Linux betrieben werden. Durch die Verwendung eines 
reduzierten Technologie-Stacks wird eine langfristige Wartbarkeit gewährleistet, während 
bewusst auf komplexe Frameworks verzichtet wird, um Transparenz zu erhöhen. Solide Tests 
und Continuous Integration werden eingesetzt, um die Softwarequalität sicherzustellen und 
die Zusammenarbeit zwischen Unix-Administratoren und Java-Entwicklern zu verbessern.
Durch dieses Konzept wird eine effiziente und skalierbare Impfregistrierung 
gewährleistet, die Ausfälle vermeidet und eine hohe Benutzerakzeptanz sichert. Die 
schnelle Reaktionsfähigkeit auf Systemänderungen und Lastspitzen stellt eine weitere 
Stärke dar. Zudem bietet das System eine nachhaltige und wiederverwendbare Lösung, die 
auch in anderen Kommunen Anwendung finden kann. Damit wird eine zuverlässige technische 
Grundlage für künftige Impfkampagnen geschaffen.
