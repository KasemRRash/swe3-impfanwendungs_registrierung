\section*{Anfragenverarbeitung}
Das Backend basiert auf einer Servlet-Architektur, die Anfragen aus dem Frontend 
entgegennimmt, verarbeitet und die entsprechenden Daten zurückliefert. Sobald ein Nutzer 
beispielsweise einen Termin buchen möchte, sendet das Frontend eine HTTP-Anfrage an das 
Backend. Diese wird von einem passenden Servlet empfangen und verarbeitet. Wenn der 
Nutzer weniger als vier Buchungen hat, wird die neue Buchung gespeichert und an das 
Frontend zurückgemeldet.
Die Architektur des Backends ist in mehrere komplementäre Module unterteilt:
\begin{itemize}
    \item Servlets: Nehmen Anfragen aus dem Frontend entgegen, verarbeiten sie und leiten sie an die passenden Komponenten weiter.
    \item Skripte für Datenbankabfragen: Enthalten SQL-Abfragen, um strukturierte Datenbankinteraktionen zu ermöglichen.
    \item Utility-Klassen: Stellen allgemeine Funktionen bereit, die wiederverwendet werden können, wie etwa Passwort-Hashing.
\end{itemize}
Durch diese klare Trennung der Verantwortlichkeiten bleibt der Code modular und gut 
wartbar. Änderungen an einzelnen Komponenten beeinflussen nicht das gesamte System, 
wodurch die Stabilität und Skalierbarkeit der Anwendung gewährleistet wird. Die klare
Struktur erleichterte zudem die Zusammenarbeit im Team, insbesondere für Mitglieder mit 
unterschiedlichem Erfahrungsniveau. Gegen Ende des Projekts zeigte sich jedoch, dass 
diese Arbeitsweise auch Performance-Nachteile haben kann – mehr dazu auf Poster Drei. 
