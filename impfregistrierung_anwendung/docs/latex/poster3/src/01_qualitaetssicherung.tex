\section*{Qualitätssicherung}

Qualitätssicherung ist ein zentraler Bestandteil in der Softwareentwicklung, um sicherzustellen, dass das System zuverlässig, sicher und effizient arbeitet. Sie hilft dabei, Fehler frühzeitig zu erkennen und zu beseitigen sowie Risiken zu minimieren.
Für Stabilität und Sicherheit sorgen wir durch systematische Tests, Versionskontrolle mittels Git und automatisierte Prozesse (DevOps). 
Wir führen unterschiedliche Tests durch, um Fehler frühzeitig zu entdecken und zu beheben.
Mittels Git-Versionierung und CI/CD-Pipelines automatisieren wir den Integrations- und Deploymentprozess, wodurch die Software stets stabil und schnell bereitgestellt werden kann. Durch Lasttests überprüfen wir die Leistungsfähigkeit und Skalierbarkeit des Systems unter verschiedenen Bedingungen. Ergebnisse werden ausgewertet und optimiert, um die Systemstabilität und Sicherheit langfristig zu gewährleisten.

