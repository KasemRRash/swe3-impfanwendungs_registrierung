\section*{Testfälle}

\FloatBarrier % Verhindert, dass die Tabelle verschoben wird

\begin{table}[H] % Nutze H statt h!
    \centering
   % \tiny
    \scriptsize
    \renewcommand{\arraystretch}{1.2}
    \begin{tabular}{|p{3cm}|p{7.9cm}|p{6.7cm}|}
        \hline
        \textbf{Testfall} & \textbf{Eingabe} & \textbf{Erwartetes Ergebnis} \\ \hline

        Registrierung (erfolgreich) &
        \parbox{6cm}{
          \texttt{curl -X POST -d} \\
          \texttt{"email=testuser@example.com\&} \\
          \texttt{password=123\&} \\
          \texttt{passwordConfirm=123"} \\
          \texttt{\$path/register}
        } &
        \texttt{"message": "Registrierung erfolgreich!", Status: success} \\ \hline

        Registrierung (Fehlgeschlagen) &
        \parbox{6cm}{
          \texttt{curl -X POST -d} \\
          \texttt{"email=testuser@example.com\&} \\
          \texttt{password=123"} \\
          \texttt{\$path/register}
        } &
        \texttt{"message": "E-Mail bereits registriert.", status: error} \\ \hline

        Login (erfolgreich) &
        \parbox{6cm}{
          \texttt{curl -X POST -c cookie -b cookie -d} \\
          \texttt{"email=testuser@example.com\&} \\
          \texttt{password=123"} \\
          \texttt{\$path/UserAnmelden}
        } &
        \texttt{"redirect": "termine", "message": "Login erfolgreich", status: success} \\ \hline

        Session prüfen &
        \parbox{6cm}{
          \texttt{curl -b cookie \$path/SessionPruefen}
        } &
        Session gültig – JSON mit Benutzerdaten oder Status \\ \hline

        Freie Slots anzeigen &
        \parbox{6cm}{
          \texttt{curl -X POST -b cookie \$path/FreieSlotsAnzeigen}
        } &
        Liste der verfügbaren Slots im JSON-Format \\ \hline

        Impfstoffe abrufen &
        \parbox{6cm}{
          \texttt{curl -b cookie} \\
          \texttt{"\$path/ImpfstoffAnzeigen?} \\
          \texttt{impfzentren=Nemerb-Nord"} \\
        } &
        Liste der Impfstoffe als JSON \\ \hline

        Termin buchen &
        \parbox{6cm}{
          \texttt{curl -X POST -b cookie -d} \\
          \texttt{"datum=2025-03-20\&zeit=08:30\&} \\
          \texttt{impfstoff\_id=1\&} \\
          \texttt{impfzentrum=Nemerb-Nord"} \\
          \texttt{\$path/TerminBuchen}
        } &
        \texttt{"message": "Termin erfolgreich gebucht", status : success} \\ \hline

        Buchungen anzeigen &
        \parbox{6cm}{
          \texttt{curl -b cookie \$path/BuchungAnzeigen}
        } &
        JSON mit allen Buchungen des Nutzers \\ \hline

        Termin stornieren &
        \parbox{6cm}{
          \texttt{curl -X POST -b cookie -d} \\
          \texttt{"buchung\_id=1"} \\
          \texttt{\$path/TerminStornieren}
        } &
        \texttt{"message": "Buchung erfolgreich gelöscht", status: success} \\ \hline

        Logout &
        \parbox{6cm}{
          \texttt{curl -X POST -b cookie \$path/UserAbmelden}
        } &
        \texttt{"message": "Abmeldung erfolgreich", status: success} \\ \hline

        Zugriff nach Logout &
        \parbox{6cm}{
          \texttt{curl -b cookie \$path/SessionPruefen}
        } &
        \texttt{"message": "Keine aktive Session", status: error} \\ \hline

    \end{tabular}
    \caption{Testfälle für die Impfregistrierungsanwendung}
    \label{tab:testfaelle}
\end{table}

\FloatBarrier % Sicherstellen, dass nichts weiter nach unten rutscht

