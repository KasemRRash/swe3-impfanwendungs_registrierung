% Beispiel Zitat
~\autocite[postnote]{kürzel}

% Beispiel für Bildintegration
\includegraphics[width=\linewidth]{src/abbildungen/fsm}
\captionof{figure}{Bildunterschrift ~\autocite[postnote]{kürzel}}

% Beispiel für eine Tabelle
\begin{longtblr}[caption={\LaTeX~Schriftgrößen}, label={schriftgrößen} ]{colspec = {X[l,m] X[2,c,m]}, rowhead=1, width=0.2\textwidth}\toprule
	Befehl                      & Ergebnis                    \\ \midrule
	\textbackslash tiny         & \tiny{winzig}               \\ \cmidrule{1-2}
	\textbackslash scriptsize   & \scriptsize{klein}          \\ \cmidrule{1-2}
	\textbackslash footnotesize & \footnotesize{etwas größer} \\ \cmidrule{1-2}
	\textbackslash small        & \small{noch etwas größer}   \\ \cmidrule{1-2}
	\textbackslash normalsize   & \normalsize{normal}         \\ \cmidrule{1-2}
	\textbackslash large        & \large{groß}                \\ \cmidrule{1-2}
	\textbackslash Large        & \Large{größer}              \\ \cmidrule{1-2}
	\textbackslash LARGE        & \LARGE{noch größer}         \\ \cmidrule{1-2}
	\textbackslash huge         & \huge{riesig}               \\ \cmidrule{1-2}
	\textbackslash Huge         & \LARGE{noch riesiger}       \\ \bottomrule
\end{longtblr}

% Beispiel für ein Listing
\begin{code}{Ein Listing}{einlisting}
	\begin{minted}{bash}
	for i in {1..10}; do 
		echo "$i"
	done	
	\end{minted}
\end{code}

% Beispiel für eine Aufzählung 
\begin{itemize}
	\item erstes Stichwort
	\item zweites Stichwort
\end{itemize}