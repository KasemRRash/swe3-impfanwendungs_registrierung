\section*{Cronjob für die Kalenderlogik}
Damit das System jederzeit aktuelle Terminverfügbarkeiten anzeigt und alte Daten nicht 
unnötig Speicherplatz belegen, wurde eine automatisierte Kalenderlogik in Form eines 
SQL-Cronjobs implementiert. Diese geplante SQL-Prozedur läuft täglich und übernimmt zwei 
Hauptaufgaben: Das Entfernen abgelaufener Buchungen des Vortags und die automatische 
Generierung neuer Zeitslots für den kommenden Tag.
Durch die Automatisierung entfällt die manuelle Terminaktualisierung, und die Datenbank 
bleibt performant, da veraltete Einträge regelmäßig entfernt werden.
Ein wesentlicher Vorteil dieser Lösung ist ihre Skalierbarkeit. Da die Terminverwaltung 
durch einen zentralen Cronjob erfolgt und die Zeitslots im Voraus generiert werden, 
bleibt das System auch bei hoher Last stabil. Nutzeranfragen greifen lediglich auf die 
bereits vorbereiteten Daten zu, wodurch keine zusätzliche Last entsteht.
Darüber hinaus können neue Impfzentren problemlos hinzugefügt werden, ohne dass 
Anpassungen an der Buchungslogik notwendig sind. Die tägliche Aktualisierung der Termine 
gewährleistet, dass das System stets aktuelle und korrekte Daten liefert, während der 
Wartungsaufwand auf ein Minimum reduziert bleibt.
